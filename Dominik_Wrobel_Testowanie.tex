\documentclass[a4paper,15pt]{article}
\usepackage{amssymb}
\usepackage{amsmath}
\usepackage[english, polish]{babel}
\usepackage[utf8]{inputenc}   % lub utf8
\usepackage[T1]{fontenc}
\usepackage{graphicx}
\usepackage{anysize}
\usepackage{enumerate}
\usepackage{times}
\usepackage{caption}
\usepackage{titlesec}
\usepackage{float}
\usepackage{titleps,kantlipsum}
\usepackage{listings}
\usepackage{xcolor}
\usepackage{hyperref}
\usepackage{framed}
\usepackage{tcolorbox}
\usepackage{mdframed}
\lstloadlanguages{Matlab}
 
\usepackage[justification=centering]{caption}
\titlelabel{\thetitle.\quad}

\pagenumbering{arabic}

\DeclareCaptionFont{white}{\color{white}}
\DeclareCaptionFormat{listing}{%
  \parbox{\textwidth}{\colorbox{darkgreen}{\parbox{\textwidth}{#1#2#3}}\vskip-4pt}}
\captionsetup[lstlisting]{format=listing,labelfont=white,textfont=white}
\lstset{frame=lrb,xleftmargin=\fboxsep,xrightmargin=-\fboxsep}

% Definicja nowego stylu strony
\newpagestyle{mypage}
{
  \headrule
  
  \sethead
  { \MakeUppercase{\thesection\quad \sectiontitle} } 
  {}
  {\thesubsection\quad \subsectiontitle}
  
  \setfoot
  {}
  {}
  {\thepage}
}

\newpagestyle{mypage_1}
{
	%\headrule
	
	
	\setfoot
	{}
	{\thepage}
	{}
}

\settitlemarks{section,subsection,subsubsection}

\pagestyle{mypage_1}

\newcommand{\ask}[2]{
    \begin{tcolorbox}[colback=black!5!white,colframe=gray,title={Pytanie #1}]
        #2
    \end{tcolorbox}
}

\newcommand{\ex}[2]{
    \begin{tcolorbox}[colback=black!5!white,colframe=black,title={Zadanie #1}]
        #2
    \end{tcolorbox}
}

%\marginsize{left}{right}{top}{bottom}
\marginsize{3cm}{3cm}{3cm}{3cm}
\sloppy
\titleformat{\section}
  {\normalfont\Large\bfseries}{\thesection}{1em}{}[{\titlerule[0.8pt]}]
 
 \definecolor{darkred}{rgb}{0.9,0,0}
\definecolor{grey}{rgb}{0.4,0.4,0.4}
\definecolor{orange}{rgb}{1,0.6,0.05}
\definecolor{darkgreen}{rgb}{0.2,0.5,0.05}
 
\definecolor{mGreen}{RGB}{2,217,39}
\definecolor{mGray}{rgb}{0.5,0.5,0.5}
\definecolor{mPurple}{RGB}{223,32,214}
\definecolor{mKeyword}{RGB}{204,152,15}
\definecolor{backgroundColour}{RGB}{68,68,68}
\definecolor{commentColor}{RGB}{243,253,254}


\lstdefinestyle{Ada}{
    backgroundcolor=\color{backgroundColour},   
    commentstyle=\color{commentColor},
    keywordstyle=\color{mKeyword},
	basicstyle=\color{mGreen}\footnotesize,    
    numberstyle=\tiny\color{mPurple},
    stringstyle=\color{mPurple},
    breakatwhitespace=false,         
    breaklines=true,                 
    %captionpos=b,                    
    keepspaces=true,                 
    numbers=left,                    
    numbersep=5pt,                  
    showspaces=false,                
    showstringspaces=false,
    showtabs=false,                  
    tabsize=2,
    language=C
}

\lstset{style=Ada}

\newcommand{\Hilight}{\makebox[0pt][l]{\color{cyan}\rule[-4pt]{0.65\linewidth}{14pt}}}

\setcounter{tocdepth}{2}

\usepackage{array}
\newcolumntype{L}[1]{>{\raggedright\let\newline\\\arraybackslash\hspace{0pt}}m{#1}}
\newcolumntype{C}[1]{>{\centering\let\newline\\\arraybackslash\hspace{0pt}}m{#1}}
\newcolumntype{R}[1]{>{\raggedleft\let\newline\\\arraybackslash\hspace{0pt}}m{#1}}

\begin{document}

\begin{titlepage}
   \begin{center}
       \vspace*{1cm}
 
       \LARGE{\textbf{Plan testów}}
 
       \vspace{0.5cm}
       \textit{Aplikacja dla biura turystycznego}
 
       \vspace{1.5cm}
 		
 	   Wersja 1.0.2 \\
       \textbf{\today}
 
       \vfill
 
       %A thesis presented for the degree of\\
       %Doctor of Philosophy
 
       \vspace{0.8cm}
 
 		Autor \\
       Dominik Wróbel\\
       
       
   \end{center}
\end{titlepage}


\newpage
\tableofcontents



\newpage
\section{Historia wersji}

\begin{center}
 \begin{tabular}{||c c c c||} 
 \hline
 Data & Wersja & Opis & Autor \\ [0.5ex] 
 \hline\hline
 14.01.2020 & 1.0.0 & Pierwsza wersja dokumentu & Dominik Wróbel \\ 
 \hline
 15.01.2020 & 1.0.0 & Testy jednostkowe i integracyjne & Dominik Wróbel \\
 \hline
 16.01.2020 & 2.0.0 & Testy jednostkowe, integracyjne i systemowe & Dominik Wróbel \\ [1ex] 
 \hline
\end{tabular}
\end{center}

\section{Wstęp}

Celem tego dokumentu jest opis przypadków testowych dla systemu biura turystycznego \textit{DreamTrip} (\url{https://github.com/software123456789/travel-agency}). 

\section{Opis systemu}
System agencji \textit{DreamTrip} to aplikacja webowa wykonana w architekturze \textit{Single Page Application} w technologii Typesript i Angular. 
\subsection{Najważniejsze funkcje systemu i role użytkowników} 

\begin{itemize}
\item Rola: klient
\begin{itemize}
\item Logowanie, rejestrowanie
\item Wyświetlanie wycieczek
\item Rezerwowanie wycieczek
\item Dodawanie wycieczek do koszyka
\item Usuwanie wycieczek z koszyka
\item Ocenianie wycieczek
\item Komentowanie wycieczek
\end{itemize} 
\item Rola: admin
\begin{itemize}
\item Dodawanie wycieczek
\item Usuwanie wycieczek
\end{itemize} 
\end{itemize}

\section{Zakres dokumentu}
W zakres tego dokumentu wchodzi opis testów:
\begin{itemize}
\item testy jednostkowe 
\item testy integracyjne
\item testy systemowe
\end{itemize}  
Dokument zawiera również analizę raportu z testów oraz zalecenia dla zespołu developerskiego.


\section{Środowisko testowe}

\subsection{Dane ogólne}
\begin{itemize}
\item System operacyjny: Windows 10
\item Baza danych: Firebase \url{https://console.firebase.google.com/u/1/project/travel-agency-ddc82/overview}
\item Przeglądarka: Google Chrome 79
\item Środowisko programistyczne: Visual Studio Code 1.41
\item Server: Node JS v12.13.0
\item npm: v6.12.0
\item Angular 8
\end{itemize}

\subsection{Biblioteki i narzędzia}
\begin{itemize}
\item Jasmine - framework javascript umożliwiający pisanie testów jednostkowych i integracyjnych
\item Karma - umożliwia wykonywanie testów, połączenie z przeglądarką
\item Protractor - framework pozwalający na pisanie testów systemowych
\end{itemize}

\newpage
\section{Testy integracyjne - przypadki testowe}







\begin{framed}
\subsection{TC.1.1 Wyświetlanie panelu admina (admin)}

\vspace{0.5cm}

\subsubsection{Warunki początkowe}
\begin{itemize}
\item Zalogowany użytkownik z rolą admin
\end{itemize}

\subsubsection{Podsumowanie}
Test ma sprawdzić czy moduł admina działa poprawnie w zakresie wyświetlania informacji raz komunikacji z serwisami zasilającymi go w dane.

\subsubsection{Kroki}
\begin{enumerate}
\item Admin przechodzi do panelu admina
\end{enumerate}

\subsubsection{Dane testowe}

\begin{center}
\begin{tabular}{ |C{1.5cm}|C{3cm}|C{4cm}|C{4cm}| } 
 \hline
 \textbf{Id} & \textbf{Dane} & \textbf{Oczekiwanie} & \textbf{Uzasadnienie} \\ \hline
 TC.1.1.1 & Poprawna lista wycieczek & Panel admina poprawnie wyświetla wycieczki & - \\ \hline
\end{tabular}
\end{center}
\end{framed}








\newpage
\begin{framed}
\subsection{TC.1.2 Usuwanie wycieczki (admin)}

\vspace{0.5cm}

\subsubsection{Warunki początkowe}
\begin{itemize}
\item Zalogowany użytkownik z rolą admin
\end{itemize}

\subsubsection{Podsumowanie}
Test ma sprawdzić czy admin może poprawnie usunąć wycieczkę z systemu. Test sprawdza interakcję komponentów panelu admina oraz serwisu zasilającego te komponenty w dane. Komponent odpowiedzialny za usuwanie danych ma emitować event w przypadku wybrania wycieczki oraz pobierać dane o wycieczce od komponentu rodzica.  

\subsubsection{Kroki}
\begin{enumerate}
\item Admin wyświetla panel admina
\item Admin usuwa wycieczkę poprzez wybranie wycieczki z tabeli
\end{enumerate}

\subsubsection{Dane testowe}

\begin{center}
\begin{tabular}{ |C{1.5cm}|C{3cm}|C{4cm}|C{4cm}| } 
 \hline
 \textbf{Id} & \textbf{Dane} & \textbf{Oczekiwanie} & \textbf{Uzasadnienie} \\ \hline
 TC.1.2.1 & Lista poprawnych wycieczek & Usunięcie z listy wybranej wycieczki & - \\ \hline
\end{tabular}
\end{center}






\newpage
\end{framed}











\newpage
\begin{framed}
\subsection{TC.1.3 Dodawanie wycieczki (admin)}

\vspace{0.5cm}

\subsubsection{Warunki początkowe}
\begin{itemize}
\item Zalogowany użytkownik z rolą admin
\end{itemize}

\subsubsection{Podsumowanie}
Test ma sprawdzić czy admin może poprawnie dodać wycieczkę do systemu. Test skupia się na testowaniu poprawności danych wprowadzanych przez użytkownika oraz sposobie obsługi tych danych przez system.
Wymagania wobec danych:
\begin{itemize}
\item name, minimum 3 znaki, maksimum 200, pole wymagane
\item country, minimum 3 znaki, maksimum 200, pole wymagane
\item cena, minimum 1, maksimum 1000000, pole wymagane, liczba
\item startDate, data w formacie dd/mm/rrrr, pole wymagane
\item endDate, data w formacie dd/mm/rrrr, pole wymagane
\item maxAvailableTrips, minimum 1, maksimum 200, pole wymagane, liczba
\item tripsDescription, minimum 10, maksimum 500, pole wymagane
\end{itemize}

\subsubsection{Kroki}
\begin{enumerate}
\item Admin wyświetla panel admina
\item Admin wypełnia formularz i dodaje wycieczkę
\end{enumerate}

\newpage
\subsubsection{Dane testowe}

\begin{center}
\begin{tabular}{ |C{1.5cm}|C{3cm}|C{4cm}|C{4cm}| } 
 \hline
 \textbf{Id} & \textbf{Dane} & \textbf{Oczekiwanie} & \textbf{Uzasadnienie} \\ \hline
 TC.1.3.1 & Brak danych & Walidacja zawodzi, przycisk do wysyłania deaktywowany & Testowanie zachowania przy braku wprowadzonych danych \\ \hline
 TC.1.3.2 & Poprawne dane & Walidacja pomyślna, przycisk do wysyłania aktywny & Testowanie zachowania przy braku wprowadzonych danych \\ \hline
 TC.1.3.3 & Data w niepoprawnym formacie & Walidacja niepomyślna, komunikat o niepoprawnej dacie & Test formatu danych \\ \hline
 TC.1.3.4 & Przekroczenie maksymalnej liczby znaków & Walidacja niepomyślna, komunikat o wymaganej liczbie znaków & Testowanie walidatora odpowiedzialnego za długość danych \\ \hline
 TC.1.3.5 & Zbyt mała liczba & Walidacja niepomyślna, komunikat o błędnej wartości & Testowanie walidatora odpowiedzialnego za zakres liczb \\ \hline
 TC.1.3.6 & Zbyt duża liczba & Walidacja niepomyślna, komunikat o błędnej wartości & Testowanie walidatora odpowiedzialnego za zakres liczb \\ \hline
 TC.1.3.7 & Znaki zamiast liczb & Walidacja niepomyślna, komunikat o błędnych danych & Testowanie walidatora odpowiedzialnego za liczby \\ \hline
 TC.1.3.8 & Wartości graniczne & Walidacja niepomyślna, komunikat o błędnych danych & Testowanie działania dla wartości granicznych walidatorów \\ \hline
\end{tabular}
\end{center}

\end{framed}









\newpage
\begin{framed}
\subsection{TC.1.4 Tworzenie paska menu dla niezalogowanego użytkownika}

\vspace{0.5cm}

\subsubsection{Warunki początkowe}
\begin{itemize}
\item Użytkownik nie jest zalogowany
\end{itemize}

\subsubsection{Podsumowanie}
Test ma sprawdzić czy menu paska nawigacyjnego widoczne w sekcji \textit{header} jest poprawnie tworzone dla niezalogowanego użytkownika, tj. zawiera przyciski do logowania i rejestracji. Test sprawdza integrację komponentu nagłówka z serwisem autoryzacji. 

\subsubsection{Kroki}
\begin{enumerate}
\item Użytkownik otwiera stronę domową aplikacji
\end{enumerate}

\subsubsection{Dane testowe}

\begin{center}
\begin{tabular}{ |C{1.5cm}|C{3cm}|C{4cm}|C{4cm}| } 
 \hline
 \textbf{Id} & \textbf{Dane} & \textbf{Oczekiwanie} & \textbf{Uzasadnienie} \\ \hline
 TC.1.4.1 & - & Lista menu wyświetla przyciski do logowania i rejestracji & - \\ \hline
\end{tabular}
\end{center}

\end{framed}




\newpage
\begin{framed}
\subsection{TC.1.5 Tworzenie paska menu dla użytkownika zalogowanego z rolą klienta}


\vspace{0.5cm}

\subsubsection{Warunki początkowe}
\begin{itemize}
\item Użytkownik nie jest zalogowany
\end{itemize}

\subsubsection{Podsumowanie}
Test ma sprawdzić czy menu paska nawigacyjnego widoczne w sekcji \textit{header} jest poprawnie wyświetlane dla zalogowanego użytkownika z rolą klienta, tj. zawiera przyciski do wyświetlania wycieczek, koszyka, historii i wylogowania.  Test sprawdza także integrację komponentu nagłówka z serwisem autoryzacji. 

\subsubsection{Kroki}
\begin{enumerate}
\item Użytkownik przechodzi na stronę domową
\item Użytkownik loguje się do systemu z rolą klienta
\end{enumerate}

\subsubsection{Dane testowe}

\begin{center}
\begin{tabular}{ |C{1.5cm}|C{3cm}|C{4cm}|C{4cm}| } 
 \hline
 \textbf{Id} & \textbf{Dane} & \textbf{Oczekiwanie} & \textbf{Uzasadnienie} \\ \hline
 TC.1.5.1 & Model użytkownika z rolą klienta i poprawnym emailem oraz hasłem & Wyświetlenie 4 przycisków w pasku menu & Po poprawnym zalogowaniu dla użytkownika z rolą klient widoczne są 4 przyciski \\ \hline
\end{tabular}
\end{center}

\end{framed}





\newpage
\begin{framed}
\subsection{TC.1.6 Tworzenie paska menu dla użytkownika zalogowanego z rolą admina}


\vspace{0.5cm}

\subsubsection{Warunki początkowe}
\begin{itemize}
\item Użytkownik nie jest zalogowany
\end{itemize}

\subsubsection{Podsumowanie}
Test ma sprawdzić czy menu paska nawigacyjnego widoczne w sekcji \textit{header} jest poprawnie wyświetlane dla zalogowanego użytkownika z rolą admina, tj. zawiera przyciski do wyświetlania wycieczek, koszyka, historii, wylogowania oraz dodatkowo panel admina.  Test sprawdza także integrację komponentu nagłówka z serwisem autoryzacji. 

\subsubsection{Kroki}
\begin{enumerate}
\item Użytkownik przechodzi na stronę domową
\item Użytkownik loguje się do systemu z rolą admina
\end{enumerate}

\subsubsection{Dane testowe}

\begin{center}
\begin{tabular}{ |C{1.5cm}|C{3cm}|C{4cm}|C{4cm}| } 
 \hline
 \textbf{Id} & \textbf{Dane} & \textbf{Oczekiwanie} & \textbf{Uzasadnienie} \\ \hline
 TC.1.6.1 & Model użytkownika z rolą admina i poprawnym emailem oraz hasłem & Wyświetlenie 5 przycisków w pasku menu & Po poprawnym zalogowaniu dla użytkownika z rolą admina widoczne jest 5 przycisków \\ \hline
\end{tabular}
\end{center}

\end{framed}







\newpage
\begin{framed}
\subsection{TC.1.7 Obliczanie minimalnej i maksymalnej ceny wycieczek}


\vspace{0.5cm}

\subsubsection{Warunki początkowe}
\begin{itemize}
\item Użytkownik jest zalogowany w systemie
\end{itemize}

\subsubsection{Podsumowanie}
Test ma sprawdzić czy najmniejsza i największa cena wycieczek, które komponent otrzymuje z serwisu jest poprawnie wyznaczana. Na podstawie tych wartości są dodawane style do kart wycieczek. 

\subsubsection{Kroki}
\begin{enumerate}
\item Użytkownik wyświetla wszystkie oferty wycieczek.
\end{enumerate}

\subsubsection{Dane testowe}

\begin{center}
\begin{tabular}{ |C{1.5cm}|C{3cm}|C{4cm}|C{4cm}| } 
 \hline
 \textbf{Id} & \textbf{Dane} & \textbf{Oczekiwanie} & \textbf{Uzasadnienie} \\ \hline
 TC.1.7.1 & Lista poprawnych wycieczek & Obliczenie największej i najmniejszej ceny wycieczki & - \\ \hline
\end{tabular}
\end{center}

\end{framed}





\newpage
\begin{framed}
\subsection{TC.1.8 Obliczanie liczby wszystkich zarezerwowanych wycieczek}


\vspace{0.5cm}

\subsubsection{Warunki początkowe}
\begin{itemize}
\item Użytkownik jest zalogowany w systemie
\end{itemize}

\subsubsection{Podsumowanie}
Test ma sprawdzić czy poprawnie obliczana jest liczba wszystkich zarezerwowanych wycieczek. Na podstawie tych wartości do strony dodawane są style. 

\subsubsection{Kroki}
\begin{enumerate}
\item Użytkownik wyświetla wszystkie oferty wycieczek.
\end{enumerate}

\subsubsection{Dane testowe}

\begin{center}
\begin{tabular}{ |C{1.5cm}|C{3cm}|C{4cm}|C{4cm}| } 
 \hline
 \textbf{Id} & \textbf{Dane} & \textbf{Oczekiwanie} & \textbf{Uzasadnienie} \\ \hline
 TC.1.8.1 & Lista poprawnych wycieczek & Obliczenie liczby wszystkich zarezerwowanych wycieczek & - \\ \hline
\end{tabular}
\end{center}

\end{framed}





\newpage
\begin{framed}
\subsection{TC.1.9 Wyświetlenie panelu filtrowania wycieczek}


\vspace{0.5cm}

\subsubsection{Warunki początkowe}
\begin{itemize}
\item Użytkownik jest zalogowany w systemie
\end{itemize}

\subsubsection{Podsumowanie}
Test ma sprawdzić czy komponent wycieczek poprawnie wyświetla panel do filtrowania wycieczek po kliknięciu na przycisk pokazania panelu. 

\subsubsection{Kroki}
\begin{enumerate}
\item Użytkownik wyświetla wszystkie oferty wycieczek
\item Użytkownik otwiera panel filtrowania wycieczek
\end{enumerate}

\subsubsection{Dane testowe}

\begin{center}
\begin{tabular}{ |C{1.5cm}|C{3cm}|C{4cm}|C{4cm}| } 
 \hline
 \textbf{Id} & \textbf{Dane} & \textbf{Oczekiwanie} & \textbf{Uzasadnienie} \\ \hline
 TC.1.9.1 & - & Wyświetlenie panelu filtrowania wycieczek & - \\ \hline
\end{tabular}
\end{center}

\end{framed}





\newpage
\begin{framed}
\subsection{TC.1.10 Dodawanie wycieczki do koszyka i aktualizacja zarezerwowanych wycieczek}


\vspace{0.5cm}

\subsubsection{Warunki początkowe}
\begin{itemize}
\item Użytkownik jest zalogowany w systemie
\item Oferta wycieczki jest dostępna do zarezerwowania
\end{itemize}

\subsubsection{Podsumowanie}
Test sprawdza czy oferta wycieczki została poprawnie dodana do koszyka, w koszyku przechowywana jest liczba ofert dla danej wycieczki, dodanie dwa razy tej samej oferty powinno zwiększyć liczbę rezerwacji dla danej oferty, ale pozostawić tylko jedną ofertę dla wycieczki w koszyku. Dodanie wycieczki do koszyka powinno spowodować aktualizację liczby zarezerwowanych wycieczek.

\subsubsection{Kroki}
\begin{enumerate}
\item Użytkownik wyświetla wszystkie oferty wycieczek
\item Użytkownik dodaje wybraną wycieczkę do koszyka
\end{enumerate}

\subsubsection{Dane testowe}

\begin{center}
\begin{tabular}{ |C{1.5cm}|C{3cm}|C{4cm}|C{4cm}| } 
 \hline
 \textbf{Id} & \textbf{Dane} & \textbf{Oczekiwanie} & \textbf{Uzasadnienie} \\ \hline
 TC.1.10.1 & Dwie takie same wycieczki & Dodanie jednej wycieczki do koszyka & Dwie takie same wycieczki powinny sprawić, że w koszyku będzie przechowywana tylko jedna z liczbą rezerwacji równą 2 \\ \hline
  TC.1.10.2 & Dwie różne wycieczki & Dodanie dwóch wycieczek do koszyka & Dwie różne wycieczki powinny sprawić, że w koszyku będą przechowywane dwie wycieczki \\ \hline
\end{tabular}
\end{center}

\end{framed}





\newpage
\begin{framed}
\subsection{TC.1.11 Usunięcie wycieczki z koszyka i aktualizacja zarezerwowanych wycieczek}


\vspace{0.5cm}

\subsubsection{Warunki początkowe}
\begin{itemize}
\item Użytkownik jest zalogowany w systemie
\item Użytkownik dodał przynajmniej jedną ofertę do koszyka
\end{itemize}

\subsubsection{Podsumowanie}
Test sprawdza czy oferta wycieczki została poprawnie dodana do koszyka, w koszyku przechowywana jest liczba ofert dla danej wycieczki, dodanie dwa razy tej samej oferty powinno zwiększyć liczbę rezerwacji dla danej oferty, ale pozostawić tylko jedną ofertę dla wycieczki w koszyku. Usunięcie wycieczki z koszyka powinno spowodować aktualizację liczby zarezerwowanych wycieczek.

\subsubsection{Kroki}
\begin{enumerate}
\item Użytkownik wyświetla wszystkie oferty wycieczek
\item Użytkownik dodaje wybraną wycieczkę do koszyka
\end{enumerate}

\subsubsection{Dane testowe}

\begin{center}
\begin{tabular}{ |C{1.5cm}|C{3cm}|C{4cm}|C{4cm}| } 
 \hline
 \textbf{Id} & \textbf{Dane} & \textbf{Oczekiwanie} & \textbf{Uzasadnienie} \\ \hline
 TC.1.11.1 & Wycieczka, której nie ma w koszyku & Brak reakcji ze strony koszyka & Aplikacja powinna być zabezpieczona przed nieoczekiwanymi danymi \\ \hline
 TC.1.11.2 & Wycieczka, która jest w koszyku & Usunięcie wycieczki z koszyka & Normalne działania aplikacji \\ \hline
\end{tabular}
\end{center}

\end{framed}







\newpage
\begin{framed}
\subsection{TC.1.12 Filtrowanie wycieczek}


\vspace{0.5cm}

\subsubsection{Warunki początkowe}
\begin{itemize}
\item Użytkownik jest zalogowany w systemie
\end{itemize}

\subsubsection{Podsumowanie}
Test sprawdza czy prezentowane dla użytkownika wycieczki są zgodne z kryteriami zaznaczonymi przez użytkownika. 

\subsubsection{Kroki}
\begin{enumerate}
\item Użytkownik wyświetla wszystkie oferty wycieczek
\item Użytkownik otwiera panel do ustawiania preferencji
\item Użytkownik wybiera preferencje
\item Użytkownik akceptuje wprowadzone preferencje
\end{enumerate}

\subsubsection{Dane testowe TODO}

\begin{center}
\begin{tabular}{ |C{1.5cm}|C{3cm}|C{4cm}|C{4cm}| } 
 \hline
 \textbf{Id} & \textbf{Dane} & \textbf{Oczekiwanie} & \textbf{Uzasadnienie} \\ \hline
 TC.1.12.1 & Filtrowanie po cenie, cena minimalna 1400, cena maksymalna 5000 & System wyświetla wycieczki w zadanym przedziale cenowym & Testowanie jednej z opcji filtrowania \\ \hline
 TC.1.12.2 & Filtrowanie po dacie, od daty 12/01/2019, do daty 01/01/2020 & System wyświetla wycieczki w zadanym przedziale czasowym & Testowanie jednej z opcji filtrowania \\ \hline
 TC.1.12.3 & Filtrowanie po kraju, wycieczki do kraju 'Poland' & System wyświetla wycieczki do zadanego kraju & Testowanie jednej z opcji filtrowania \\ \hline
 TC.1.12.4 & Filtrowanie po ratingu, minimalny rating równy 3.0 & System wyświetla wycieczki z ratingiem większym niż zadany & Testowanie jednej z opcji filtrowania \\ \hline
\end{tabular}
\end{center}

\end{framed}


\newpage
\section{Testy systemowe - przypadki testowe}

\begin{framed}
\subsection{TC.2.1 Wyświetlanie wiadomości powitalnej}


\vspace{0.5cm}

\subsubsection{Warunki początkowe}
\begin{itemize}
\item Brak
\end{itemize}

\subsubsection{Podsumowanie}
Test ma sprawdzić czy wiadomość powitalna na stronie domowej wyświetla się poprawnie.

\subsubsection{Kroki}
\begin{enumerate}
\item Użytkownik przechodzi na stronę domową
\end{enumerate}

\subsubsection{Dane testowe}

\begin{center}
\begin{tabular}{ |C{1.5cm}|C{3cm}|C{4cm}|C{4cm}| } 
 \hline
 \textbf{Id} & \textbf{Dane} & \textbf{Oczekiwanie} & \textbf{Uzasadnienie} \\ \hline
 TC.2.1.1 & - & Wyświetlenie wiadomości powitalnej & - \\ \hline
\end{tabular}
\end{center}

\end{framed}



\newpage
\begin{framed}
\subsection{TC.2.2 Wyświetlenie menu dla niezalogowanego użytkownika}

\vspace{0.5cm}

\subsubsection{Warunki początkowe}
\begin{itemize}
\item Użytkownik nie jest zalogowany
\end{itemize}

\subsubsection{Podsumowanie}
Test ma sprawdzić czy linki dla niezalogowanego użytkownika są wyświetlanie poprawnie.

\subsubsection{Kroki}
\begin{enumerate}
\item Użytkownik przechodzi na stronę domową
\end{enumerate}

\subsubsection{Dane testowe}

\begin{center}
\begin{tabular}{ |C{1.5cm}|C{3cm}|C{4cm}|C{4cm}| } 
 \hline
 \textbf{Id} & \textbf{Dane} & \textbf{Oczekiwanie} & \textbf{Uzasadnienie} \\ \hline
 TC.2.2.1 & - & Wyświetlenie w menu linku do logowania i zakładania konta & - \\ \hline
\end{tabular}
\end{center}

\end{framed}



\newpage
\begin{framed}
\subsection{TC.2.3 Logowanie użytkownika z rolą klienta i dostosowanie paska menu}


\vspace{0.5cm}

\subsubsection{Warunki początkowe}
\begin{itemize}
\item Brak
\end{itemize}

\subsubsection{Podsumowanie}
Test ma sprawdzić czy system poprawnie loguje użytkownika oraz jak reaguje w przypadku nieprawidłowych danych logowania. Oprócz tego po poprawnym zalogowaniu system powinien dostosować pasek menu do użytkownika i jego roli. 

\subsubsection{Kroki}
\begin{enumerate}
\item Użytkownik loguje się do systemu niepoprawnymi danymi
\item Użytkownik loguje się do systemu poprawnymi danymi z rolą klient
\item Użytkownik przechodzi na stronę z ofertami
\end{enumerate}

\subsubsection{Dane testowe}

\begin{center}
\begin{tabular}{ |C{1.5cm}|C{3cm}|C{4cm}|C{4cm}| } 
 \hline
 \textbf{Id} & \textbf{Dane} & \textbf{Oczekiwanie} & \textbf{Uzasadnienie} \\ \hline
 TC.2.3.1 & Niepoprane dane logowania & System pozostaje na stronie domowej, brak zmiany paska menu & Testowanie działania dla niepoprawnych danych \\ \hline
 TC.2.3.2 & Poprawne dane logowania & System przechodzi na stronę z wycieczkami, dostosowanie paska menu & Testowanie działania dla poprawnych danych \\ \hline
\end{tabular}
\end{center}

\end{framed}




\newpage
\begin{framed}
\subsection{TC.2.4 Logowanie użytkownika z rolą admina i dostosowanie paska menu}


\vspace{0.5cm}

\subsubsection{Warunki początkowe}
\begin{itemize}
\item Brak
\end{itemize}

\subsubsection{Podsumowanie}
Test ma sprawdzić czy system poprawnie loguje użytkownika oraz jak reaguje w przypadku nieprawidłowych danych logowania. Oprócz tego po poprawnym zalogowaniu system powinien dostosować pasek menu do roli admina. 

\subsubsection{Kroki}
\begin{enumerate}
\item Użytkownik loguje się do systemu niepoprawnymi danymi
\item Użytkownik loguje się do systemu poprawnymi danymi z rolą admin
\item Użytkownik przechodzi na stronę z ofertami
\end{enumerate}

\subsubsection{Dane testowe}

\begin{center}
\begin{tabular}{ |C{1.5cm}|C{3cm}|C{4cm}|C{4cm}| } 
 \hline
 \textbf{Id} & \textbf{Dane} & \textbf{Oczekiwanie} & \textbf{Uzasadnienie} \\ \hline
 TC.2.3.1 & Niepoprane dane logowania & System pozostaje na stronie domowej, brak zmiany paska menu & Testowanie działania dla niepoprawnych danych \\ \hline
 TC.2.3.2 & Poprawne dane logowania & System przechodzi na stronę z wycieczkami, dostosowanie paska menu & Testowanie działania dla poprawnych danych \\ \hline
\end{tabular}
\end{center}

\end{framed}





\newpage
\begin{framed}
\subsection{TC.2.5 Wyświetlanie kryteriów filtrowania}


\vspace{0.5cm}

\subsubsection{Warunki początkowe}
\begin{itemize}
\item Użytkownik jest zalogowany w systemie
\end{itemize}

\subsubsection{Podsumowanie}
Test ma sprawdzić czy system poprawnie prezentuje panel do filtrowania wycieczek. 

\subsubsection{Kroki}
\begin{enumerate}
\item Użytkownik przechodzi na stronę z ofertami wycieczek
\item Użytkownik wybiera przycisk panelu filtrowania
\end{enumerate}

\subsubsection{Dane testowe}

\begin{center}
\begin{tabular}{ |C{1.5cm}|C{3cm}|C{4cm}|C{4cm}| } 
 \hline
 \textbf{Id} & \textbf{Dane} & \textbf{Oczekiwanie} & \textbf{Uzasadnienie} \\ \hline
 TC.2.5.1 & - & Poprawne wyświetlenie panelu filtrowania & - \\ \hline
\end{tabular}
\end{center}

\end{framed}




\newpage
\begin{framed}
\subsection{TC.2.6 Dodawania i usuwanie wycieczki z koszyka}


\vspace{0.5cm}

\subsubsection{Warunki początkowe}
\begin{itemize}
\item Użytkownik jest zalogowany w systemie
\end{itemize}

\subsubsection{Podsumowanie}
Test ma sprawdzić czy system poprawie wyświetla elementy, które służą do dodawania i usuwania elementu z koszyka. Przycisk dodawania jest dostępny po zalogowaniu o ile wycieczka jest dostępna. Przycisk usuwania jest dostępny jeśli użytkownik dodał wybraną wycieczkę do koszyka. 

\subsubsection{Kroki}
\begin{enumerate}
\item Użytkownik przechodzi na stronę z ofertami wycieczek
\item Użytkownik dodaje wycieczkę do koszyka
\item Użytkownik usuwa wycieczkę z koszyka
\end{enumerate}

\subsubsection{Dane testowe}

\begin{center}
\begin{tabular}{ |C{1.5cm}|C{3cm}|C{4cm}|C{4cm}| } 
 \hline
 \textbf{Id} & \textbf{Dane} & \textbf{Oczekiwanie} & \textbf{Uzasadnienie} \\ \hline
 TC.2.6.1 & - & System wyświetla przycisk dodawania wycieczki, system wyświetla przycisk usuwania wycieczki po dodaniu wycieczki do koszyka & - \\ \hline
\end{tabular}
\end{center}

\end{framed}




\newpage
\begin{framed}
\subsection{TC.2.7 Oznaczanie wycieczki z mała liczbą dostępnych ofert}


\vspace{0.5cm}

\subsubsection{Warunki początkowe}
\begin{itemize}
\item Użytkownik jest zalogowany w systemie
\end{itemize}

\subsubsection{Podsumowanie}
Test ma sprawdzić czy wycieczki z liczbą dostępnych ofert mniejszą niż 4 są poprawnie wyświetlane, tj. czy mają odpowiednią wartość koloru dla tła. 

\subsubsection{Kroki}
\begin{enumerate}
\item Użytkownik przechodzi na stronę z wycieczkami
\item Użytkownik dodaje do koszyka 4 z 5 dostępnych wycieczek
\end{enumerate}

\subsubsection{Dane testowe}

\begin{center}
\begin{tabular}{ |C{1.5cm}|C{3cm}|C{4cm}|C{4cm}| } 
 \hline
 \textbf{Id} & \textbf{Dane} & \textbf{Oczekiwanie} & \textbf{Uzasadnienie} \\ \hline
 TC.2.7.1 & - & Wycieczka wyświetlania jest na innym kolorze tła & - \\ \hline
\end{tabular}
\end{center}

\end{framed}


\newpage
\section{Raport z testów}
\begin{lstlisting}
ng test
...
Chrome 79.0.3945 (Windows 10.0.0): Executed 27 of 27 SUCCESS (9.044 secs / 8.962 secs)
TOTAL: 27 SUCCESS
TOTAL: 27 SUCCESS
\end{lstlisting}


\begin{lstlisting}
ng e2e
...
Executed 7 of 7 specs (2 FAILED) in 51 secs.
[18:42:16] I/launcher - 0 instance(s) of WebDriver still running
[18:42:16] I/launcher - chrome #01 failed 2 test(s)
[18:42:17] I/launcher - overall: 2 failed spec(s)
[18:42:17] E/launcher - Process exited with error code 1
\end{lstlisting}

\begin{figure}[H]
\centerline{\includegraphics[scale=0.75]{testy1.png}}
\end{figure}

\newpage
\begin{lstlisting}
ng test --watch=false --code-coverage

=============================== Coverage summary ===============================
Statements   : 37.13% ( 186/501 )
Branches     : 14.55% ( 8/55 )
Functions    : 27.37% ( 49/179 )
Lines        : 33.62% ( 155/461 )
================================================================================
\end{lstlisting}

\begin{figure}[H]
\centerline{\includegraphics[scale=0.5]{testy2.png}}
\end{figure}

Kod testów:
\begin{itemize}
\item \url{https://github.com/software123456789/travel-agency-tests.git}
\end{itemize}


\section{Zalecenia}

\begin{itemize}
\item Raport pokazuje zadowalające pokrycie testami jednostkowymi i integracyjnymi dla komponentów prezentacyjnych, zaleca się jednak wykonanie większej ilości testów dla serwisów 
\item Naprawy wymaga przycisk do wyświetlania panelu filtrowania (T.C.2.4), który poprawnie otwiera panel, ale przestaje działać po próbie ponownego otworzenia
\end{itemize}

Komentarz: mała liczba testów jednostkowych i integracyjnych dla serwisów wynika z konieczności mockowania dla tych serwisów serwisu, który zapewniany jest przez moduł angular firestore.






\end{document}
